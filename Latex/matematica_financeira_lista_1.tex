\documentclass[12pt,a4paper]{article}

\usepackage{cancel,amsmath,amssymb,amsfonts,setspace}

\begin{document}
\pagenumbering{gobble}
  \begin{center}
    \textbf{Primeira lista de Exercícios.}\\
    06 de Março, 2025.
  \end{center}
  \vspace*{.5cm}

  \textbf{1.} Quanto de juros um capital de R\$ $1.000,00 $ aplicado à taxa de
  juros simples de $10\%$ ao ano, rende ao longo de $2$ anos?
  \[
    J = C * i * n = 1.000 * 0,1 * 2 = \text{R\$}\ 200,00 
  \]

  \textbf{2.} Em quanto tempo um capital de R\$ $10.000.000,00$ aplicado a $4,3\%$ a.m. (juros
  simples) renderá R\$ $645.000,00$?
  \[
    J = C * i * n \implies 645.000 = 10.000.000 * 0,043 * n 
  \]
  Isolando n:
  \[
    n = \frac{645.000}{10.000.000 * 0,043} = 1,5\ \text{mês ou}\ 45\ \text{dias}
  \]
  
  \textbf{3.} Bruno aplicou R\$ 30.000,00 a juros simples, pelo prazo de 6 meses, e recebeu R\$
  9.000,00 de juros. Qual a taxa mensal da aplicação?
  \[
    J = C*i*n \implies 9.000 = 30.000*i*6
  \]
  Isolando i:
  \[
    i = \frac{9.000}{30.000*6} = 0,05\ \text{a.m.}
  \]
  
  \textbf{4.} Que quantia aplicada a 4\% a.m. (juros simples), durante três meses e dez dias,
  rende R\$ 28.000,00?

  Converter o tempo apenas para meses:\\
  Se manter em fração terá o resultado exato, se converter dará uma apro-ximação:
  \[
    3\ \text{meses}  + 10\ \text{dias} \implies 3 + \frac{10}{30}\ = 3 + \frac{1}{3} = \frac{9 + 1}{3} = \frac{10}{3}
  \]
  \[
    J = C*i*n \implies 28.000 = C*0,04*\frac{10}{3}
  \]
  Isolando C:
  \[
    C = \frac{28.000}{0,04*\frac{10}{3}} = \frac{28.000 * 3}{0,4} = \text{R\$}\ 210.000
  \]

  \newpage
  \textbf{5.} Qual é a taxa de juros (simples) anual cobrada em cada um dos casos abaixo, se
  uma pessoa aplicou o capital de R\$ 1.000,00 e recebeu:
  \vspace*{.5cm}

  \hspace*{.5cm} a)M = R\$ 1.420,00; n = 2 anos.
  \[
    M = C(1 + in) \implies 1.420 = 1.000(1 + i*2)
  \]
  Isolando o i:
  \[
    1 + 2i = \frac{1.420}{1.000} = 1,42 \implies 2i = 1,42 - 1 = 0,42 \implies i = \frac{0,42}{2}
  \]
  Temos que i:
  \[
   0,21\ \text{a.a.}
  \]

  \hspace*{.5cm} b) M = R\$ 1.150,00;n = 10 meses.

  Converta meses para anos:
  \[
    \frac{10}{12} = \frac{5}{6}
  \]
  \[  
    M = C(1 + in) \implies 1.150 = 1.000\left( 1 + \frac{5}{6}i\right)
  \]
  Isolando i:
  \[
    1 + \frac{5}{6}i = \frac{1.150}{1.000} = 1,15 \implies \frac{5}{6}i =1,15 - 1 = 0,15 \implies i = \frac{6*0,15}{5}
  \]
  Temos como resultado:
  \[
    i = 0,18\ \text{a.a.}
  \]

  \hspace*{.5cm} c) M = R\$ 1.350,00; n = 1 ano e 9 meses.
  Convertendo 1 ano e 9 meses apenas para anos:
  \[
    1 + \frac{9}{12} = 1 + \frac{3}{4} = \frac{7}{4}\ \text{anos}
  \]
  \[
    M = C( 1 + in ) \implies 1.350 = 1.000\left( 1 + \frac{7}{4}i\right)
  \]
  Isolando i:
  \[
    1 + \frac{7}{4} = \frac{1.350}{1.000} = 1,35 \implies \frac{7}{4} = 1,35-1 = 0,35
  \]
  \[
    i = \frac{0,35*4}{7} = 0,2\ \text{a.a.}
  \]
  \newpage
  \textbf{6.} Uma geladeira é vendida à vista por R\$ 1.500,00 ou então à prazo com R\$ 450,00
  de entrada mais uma parcela de R\$ 1.200,00 após 4 meses. Qual a taxa mensal de juros
  simples do financiamento?
  \vspace*{.5cm}

  O juros só interfere no valor excedente ao que foi pago da entrada:
  \[
    C = 1.500 - 450 = 1.050
  \]
  \[
    M = C(1 + in) \implies 1.200 = 1.050 ( 1 + 4i)
  \]
  Isolando i:
  \[
    1 + 4i = \frac{1.200}{1.050} \approx 1,143 \implies 4i = 1,143 - 1 = 0,143
  \]
  Tendo assim:
  \[
    i = \frac{0,143}{4} = 0,0357\ \text{a.m.}
  \]

  \textbf{7.} Um comerciante aceita cheque pré-datado para 30 dias, mas cobra juros compostos
  de 8\% a.m. sobre o preço à vista. Quanto custa à vista uma mercadoria que ao ser paga
  em 30 dias sai por R\$ 27,00?
  \[
    M = C ( 1 + i)^n \implies 27 = C ( 1 + 0,08 )
  \]
  Isolando C:
  \[
    C = \frac{27}{1,08} = \text{R\$}\ 25
  \]

  \textbf{8.} Um capital ficou depositado durante 10 meses à taxa de 8\% a.m. no regime de
  juros simples. Findo esse prazo, o montante auferido foi aplicado durante 15 meses a juros
  simples à taxa de 10\% a.m. Calcule o valor do capital inicial aplicado, sabendo-se que o
  montante final recebido foi de R\$ 1.125.000,00.

  \[
    \begin{array}{lr}
      10\ \text{meses} & 8\% \\
      15\ \text{meses} & 10\% 
    \end{array}
  \]

  Temos o montante final e queremos saber do valor inicial. Para tal vamos resolver em etapas,
  do último para o primeiro:
  \[
    M = C ( 1 + in ) \implies 1.125.000 = C ( 1 + 15*0,10) \implies 1.125.000 = C ( 1 + 1,5 )
  \]
  Isolando C:
  \[ 
    C = \frac{1.125.000}{2,5} = 450.000
  \]

  Com esse montante, podemos calcular o inicial:
  \[
    M = C ( 1 + in ) \implies 450.000 = C ( 1 + 0,08*10) \implies 450.000 = C ( 1 + 0,8 )
  \]
  Isolando C:
  \[
    C = \frac{450.000}{1,8} = 250.000
  \]

  O valor inicial é: R\$ 250.000  

  \textbf{9.} Se o valor atual de um título é igual a 4/5 de seu valor nominal e o prazo de
  aplicação for de 15 meses, qual a taxa de juros simples considerada?
  \[
    N = A ( 1 + in ) \implies 1 = \frac{4}{5} (1 + i*15)
  \]
  Isolando i:
  \[
    1 + 15i = {5*1}{4} \implies 15i = 1,25 - 1 \implies i = \frac{0,25}{15} = 0,01\bar{6}
  \]

  \textbf{10.} Qual o valor futuro de um capital de R\$ 2.000,00, se n = 24 meses e taxa de juros i = 2\% a.m., para:
  \hspace*{.5cm} Capitalização simples?
  \[
    N = A ( 1 + in ) = 2.000 (1 + 0,02*24) = 2.000 (1 + 0,48) = 2.000(1,48)
  \]
  \[
    N = \text{R\$}\ 2.960,00
  \]
  \hspace*{.5cm} Capitalização composta?
  \[
    N = A ( 1  + i)^n = 2.000 ( 1 + 0,02)^{24} = 2.000 (1,02)^{24} = 2.000(1,60843725)
  \]
  \[
    N = 3.216,8745 = \text{R\$}\ 3.216,87
  \]
  
  \textbf{11.} Qual o valor atual de um montante de R\$ 5.000,00, se n = 36 meses e taxa de
  juros i = 1\% a.m., para:
  \hspace*{.5cm} Capitalização simples?
  \[
    N = A ( 1 + in ) \implies 5.000  = A (1 + 0,01*36) = A (1,36)
  \]
  Isolando A:
  \[
    A = \frac{5.000}{1,36} \approx \text{R\$}\ 3.676,47
  \]
  \hspace*{.5cm} Capitalização composta?
  \[
    N = A ( 1  + i)^n \implies 5.000 = A ( 1 + 0,01)^{36} = A (1,01)^{36} = A(1,43076878)
  \]
  Isolando A:
  \[
    A = \frac{5.000}{1,43076878} \approx \text{R\$}\ 3.494,62
  \]
  \newpage

  \textbf{12.} Determine a taxa de desconto simples comercial de um título negociado 60 dias
  antes do seu vencimento, sendo seu valor de resgate igual a R\$ 26.000,00 e valor atual na
  data do desconto de R\$ 24.436,10.
  \[
    A_c = N_c - D_c \implies 24.436,10 = 26.000 ( 1 - 60i )
  \]
  Isolando i:
  \[
    1 - 60i = \frac{24.436,10}{26.000} = 0,93985
  \]
  \[
    60i = 1 - 0,93985 = 0,06015
  \]
  \[
    i = \frac{0,06015}{60} = 0,010025\ \text{a.d.}
  \]
  Ela colocou ao mês no slide, chegamos assim: \\
  $60$ dias = $2$ meses
  \[
   2i = 0,06015 \implies i = \frac{0,06015}{2} = 0,030075\ \text{a.m.}
  \]

  \textbf{13.} Uma pessoa possui 3 títulos aplicados no mercado financeiro, sendo seus valores
  de resgate: R\$ 110.000,00, R\$ 150.000,00 e R\$ 200.000,00. As datas de resgate são daqui
  a 28 dias, 47 dias e 72 dias, respectivamente. Qual o valor presente total desses títulos,
  considerando-se uma taxa de juros simples de 30\% a.a. e o regime de desconto racional?
  \vspace*{.5cm}

  Primeiramente, vamos fazer uma tabela do que precisamos, o que temos e converter o que for necessário.
  \vspace*{.5cm}

  \( \left\{
  \begin{array}{ccllr}
    V_p & = & A_1 + A_2 + A_3 \\
    \vspace*{.5cm}
    T_1 & = & \text{R\$}\ 110.000 & 28\ \text{dias} = \dfrac{28}{360}\ \text{ano} \\
    \vspace*{.5cm}
    T_2 & = & \text{R\$}\ 150.000 & 47\ \text{dias} = \dfrac{47}{360}\ \text{ano} \\
    T_3 & = & \text{R\$}\ 200.000 & 72\ \text{dias} = \dfrac{72}{360}\ \text{ano} \\    i   & = & 0,30\ \text{a.a.}
  \end{array} \right.
  \)

  \[
    A_1 = 110.000 \left( 1 - 0,3 * \frac{28}{360} \right) = 110.000 ( 0,97\bar{6}) = 107.433,33
  \]

  \[
    A_2 = 150.000 \left( 1 - 0,3 * \frac{47}{360} \right) = 150.000 ( 0,9608\bar{3}) = 144.125
  \]

  \[
    A_3 = 200.000 \left( 1 - 0,3 * \frac{72}{360} \right) = 200.000 (0,94) = 188.000
  \]
  \[
    A_3 = 188.000
  \]

  Logo temos o valor inicial aplicado:

  \[ V_p = 107.433,33 + 144.125 + 188.000 = 439.558,33 \]

  Refiz e não deu igual o dela.
  \vspace*{.5cm}

  \textbf{14.} Uma empresa retira do Banco X um empréstimo por 3 meses no valor de R\$
  500.000,00. Se a taxa de juros simples for de 26\% a.a. e, além disso, o banco cobrar 1\%
  a título de taxa de serviço, qual será o desconto comercial?
  
  \[
    D_c = N * i * n = 500.000*0,26*\frac{3}{12} = 32.500
  \]
  \[
    D_c + TS = 32.500 + 500.000 * 0,01 = 32.500 + 5.000 = \text{R\$}\ 37.500
  \]

  \textbf{15.} Numa operação de desconto de um título a vencer em 5 meses, o desconto comercial é 
  R\$ 140,00 maior que o desconto racional. Qual será o valor nominal do título, se a
  taxa de juros simples empregada nos descontos for de 24\% a.a.?
  \vspace*{.5cm}

  \(
    \left\{ 
      \begin{array}{clll}
        \vspace*{.5cm}
        D_r = \dfrac{N*i*n}{1+i*n} \\
        D_c = N*i*n
      \end{array}
    \right.
  \)

  \[
  D_c = D_r + 140
  \]

  Substituindo os valores que temos:

  \[
    N*0,24*\frac{5}{12} = \frac{N*0,24*\frac{5}{12}}{1+0,24*\frac{5}{12}} + 140
  \]

  \[
    0,1N = \frac{0,1N}{1 + 0,1} + 140 \implies 0,1N - \frac{0,1N}{1,1} = 140
  \]
  \[
    N\left( 0,1 - \frac{0,1}{1,1} \right) = 140 \implies 0,0\overline{09}N = 140 
  \]
  \[
    N = \frac{140}{0,0\overline{09}} = \text{R\$}\ 15.400
  \]

  Também não bateu com o do slide, para bater, $D_c$ deveria ser 1.400 maior que o $D_r$.

  \textbf{16.} Uma duplicata com valor de resgate igual a R\$ 1.000.000,00 é resgatada 3 meses
  antes de seu vencimento. Sabendo-se que a taxa de desconto simples comercial é de 3,5\%
  a.m. e que o IOF retido é de 0,123\% a.m., determine:

  \hspace*{.5cm} a) O valor do desconto comercial e o respectivo IOF retidos.
  \[
    D_c = Nin = 1.000.000*0,035*3 = 105.000
  \]
  \[
    IOF = A_c*i*n = (N - D_c)*i*n
  \]
  \[
    IOF = (1.000.000 - 105.000)*0,00123*3 = 3.302,55
  \]
        
  \hspace*{.5cm} b) O valor atual comercial e o valor líquido liberado.
  \[
    A_c = 1.000.000 - 105.000 = 895.000
  \]
  \[
    VL = A_c - IOF = 895.000 - 3.302,55 = 891.697,45
  \]
  
  \textbf{17.} Qual é a taxa de juros mensal recebida por um investidor que aplica R\$ 1.000,00
  e resgata os montantes, segundo as hipóteses abaixo (juros compostos):

  \hspace*{.5cm} a) R\$ 1.076,89 – 3 meses.
  \[
    M = C (1 + i)^n \implies 1.076,89 = 1.000(1 + i)^3
  \]
  Isolando i:
  \[
    (1 + i)^3 = \frac{1.076,89}{1.000} = 1,07689
  \]
  \[
    1+i = \sqrt[3]{1,07689} = 1.0249998
  \]
  \[
    i = 1.0249998 - 1 = 0,0249998
  \]

  \hspace*{.5cm} b) R\$ 1.125,51 - 4 meses.
  \[
    1.125,51 = 1.000 (1+i)^4
  \]
  \[
    (1+i)^4 = \frac{1.125,51}{1.000} = 1,12551
  \]
  \[
    1+i = \sqrt[4]{1,12551} \approx 1,03
  \]
  \[
    i = 1,03 - 1 = 0,03
  \]

  \hspace*{.5cm} c) R\$ 1.340,10 – 6 meses.
  \[
    1.340,10 = 1.000 ( 1 + i)^6
  \]
  \[
    (1 + i)^6 = \frac{1.340,10}{1.000} = 1,34010
  \]
  \[
    1 + i = \sqrt[6]{1,34010} \approx 1,05
  \]
  \[
    i = 1,05 - 1 = 0,05
  \]

  \textbf{18.} A diferença entre os descontos simples comercial e racional de um título de crédito
  pagável daqui a 4 meses, à taxa de 6\% a.m., é igual a R\$ 2.100,00. Encontre:

  a) O valor nominal.
  \[
    D_r = \frac{N*i*n}{1+i*n} \qquad D_c = N*i*n
  \]
  \[
    D_r - D_c = 2.100
  \]

  \[
     \frac{N*i*n}{1+i*n} -  N*i*n = 2.100 \implies \frac{N*0,06*4}{1+0,06*4} -  N*0,06*4 = 2.100
  \]
  \[
    \frac{N*0,24}{1+0,24} -  N*0,24 = 2.100 \implies N \left( \frac{0,24}{1,24} - 0,24 \right) = 2.100
  \]
  \[
    N = \frac{2.100}{0,046451613} \approx 45.208,33
  \]

  b) O desconto comercial.
  \[
    D_c = 45.208,33*0,24 = 10.850
  \]

  c) O desconto racional.
  \[
    D_r = \frac{45.208,33*0,24}{1,24} = 8.750
  \]

  \textbf{19.} Para cada taxa nominal apresentada a seguir, calcule a taxa 
  efetiva anual (capitalização composta):

  a) 9\% a.a. capitalizados mensalmente.
  \[
    i = \left( 1 + \frac{0,09}{12} \right)^12 - 1 \approx 0,0938
  \]

  b) 14\% a.a. capitalizados trimestralmente.
  \[
    i = \left( 1 + \frac{0,14}{4} \right)^4 - 1 \approx 0,1475
  \]
  
  c) 15\% a.a. capitalizados semestralmente.
  \[
    i = \left( 1 + \frac{0,15}{2} \right)^2 - 1 \approx 0,1556
  \]
  
  d) 12\% a.a. capitalizados anualmente
  \[
    i = \left( 1 + \frac{0,12}{1} \right)^1 - 1 = 0,12
  \]

  \textbf{20.} Um título de valor nominal de R\$ 35.000,00 é negociado mediante uma operação
  de desconto composto comercial (“por fora”) 3 meses antes do seu vencimento. A taxa
  de desconto adotada é de 5\% a.m. Determine:

  a) O valor atual comercial.
  \[
    A_c = N(1-i)^n = 35.000(1-0,05)^3 = 30.008,12
  \]

  b) O desconto comercial.
  \[
    D_c = N - A_c = 35.000 - 30.008,12 = 4.991,875
  \]

  \textbf{21.} Sabe-se que a taxa nominal de uma aplicação financeira é de 12\% a.a., com
capitalização mensal composta. Determine:

  a) Quanto valerá uma aplicação de R\$ 10.000,00 depois de 5 meses.
  \[
    M = C ( 1 + i)^n = 10.000 ( 1 + 0,1 )^5 = 10.510,10
  \]


  b) A taxa efetiva anual da aplicação financeira.
  \[
    i = \left( 1 + \frac{j}{m} \right)^m = \left( 1 + \frac{0,12}{12} \right)^12 - 1 \approx 0,1268  
  \]

  c) A taxa efetiva mensal da aplicação financeira.
  \[
    i = \frac{j}{m} = \frac{0,12}{12} = 0,01
  \]

  \textbf{22.} Um empresário possui dois títulos com valores de resgate de R\$ 50.000,00 e R\$
70.000,00, vencíveis em 3 e 7 meses, respectivamente, a partir da data presente. Sem
liquidez para quitar os débitos em suas datas, negocia com a instituição bancária – que
estipula juros compostos de 3\% a.m. – para substituição das dívidas por duas outras de
igual valor a vencerem em 9 e 12 meses. Determine o valor de cada débito nesta nova
situação.
\[
  N = A ( 1 + i )^n \implies A = \frac{N}{(1+i)^n}
\]

\[
  A_1 = \frac{50.000}{(1+0,03)^3} \approx 45.758,88
\]
\[
  A_2 = \frac{70.000}{(1+0,03)^7} \approx 56.917,08
\]
\[
  A_t = 45.758,88 + 56.917,88 = 102.675,96
\]

Com isso, temos que fazer a seguinte equação:

\[
  X \left( \frac{1}{1,03^9} + \frac{1}{1,03^{12}} \right) = 102.675,96
\]
\[
  X = \frac{102.675,96}{1,46779661} = 69.952,444
\]

Sabendo

\textbf{23.} Aplique o ”Método Hamburguês” para calcular o valor dos juros simples
e encargos incidentes sobre os saldos devedores de uma empresa portadora de 
”Cheque Especial”durante o mês de novembro, quando são cobrados juros de 4,2\% a.m. e IOF de
0,123\% a.m., conforme extrato a seguir (data da cobrança: 26/11):
\vspace*{.5cm}

Calcule os dias entre cada movimento:
 
\begin{center}
  3 dias; 7 dias; 1 dia; 10 dias; 3 dias.
\end{center}

Multiplique pelo saldo respectivo:
\begin{center}
  200.000*3 = 600.000; \\ 30.00*7 = 210.000; \\
  80.000 * 1 = 80.000; \\ 100.000*10 = 1.000.000 \\
  100.00 * 3 = 300.000
\end{center}

Calcule os juros e IOF dos depósitos:

\[
  J = \frac{0,042}{30} (210.000+80.000+300.000) \approx 826
\]
\[
  IOF = \frac{0,00123}{30} ( 210.000 + 80.000 + 300.000 ) \approx 24,19
\]


\textbf{24.} E preferível receber R\$ 8.000,00 hoje ou R\$ 10.000,00 daqui a 6 meses? Considere
juros compostos com taxa de 4\% ao mês.

Para comparar, precisamos calcular qual valor teremos:

\[
  N = A ( 1 + i )^n = 8.000 (1,04)^6 \approx 10.122,55
\]

Pelo valor ser maior, compensa receber 8.000 hoje
\vspace*{.5cm}

\textbf{25.} Um imóvel está à venda por 4 parcelas semestrais de R\$ 50.000,00, vencendo a
primeira em 6 meses. Um financista propõe a compra deste imóvel, pagando-o em duas
parcelas iguais, uma no ato da compra e outra após 1 ano. Qual é o valor das parcelas,
se a taxa de juros for de 20\% a.s. (juros compostos)?

\end{document}